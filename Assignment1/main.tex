% \iffalse
\let\negmedspace\undefined
\let\negthickspace\undefined
\documentclass[journal,12pt,onecolumn]{IEEEtran}
\usepackage{cite}
\usepackage{amsmath,amssymb,amsfonts,amsthm}
\usepackage{algorithmic}
\usepackage{graphicx}
\usepackage{textcomp}
\usepackage{xcolor}
\usepackage{txfonts}
\usepackage{listings}
\usepackage{enumitem}
\usepackage{mathtools}
\usepackage{gensymb}
\usepackage{comment}
\usepackage[breaklinks=true]{hyperref}
\usepackage{tkz-euclide} 
\usepackage{listings}
\usepackage{gvv}                                        
\def\inputGnumericTable{}                                 
\usepackage[latin1]{inputenc}                                
\usepackage{color}                                            
\usepackage{array}                                            
\usepackage{longtable}                                       
\usepackage{calc}                                             
\usepackage{multirow}                                         
\usepackage{hhline}                                           
\usepackage{ifthen}                                           
\usepackage{lscape}

\newtheorem{theorem}{Theorem}[section]
\newtheorem{problem}{Problem}
\newtheorem{proposition}{Proposition}[section]
\newtheorem{lemma}{Lemma}[section]
\newtheorem{corollary}[theorem]{Corollary}
\newtheorem{example}{Example}[section]
\newtheorem{definition}[problem]{Definition}
\newcommand{\BEQA}{\begin{eqnarray}}
\newcommand{\EEQA}{\end{eqnarray}}
\newcommand{\define}{\stackrel{\triangle}{=}}
\theoremstyle{remark}
\newtheorem{rem}{Remark}
\begin{document}

\bibliographystyle{IEEEtran}
\vspace{3cm}

\title{LIMITS,CONTINUITY}
\author{EE24BTECH11046 - NENAVATH VASU $^{*}$% <-this % stops a space
}
\maketitle
\bigskip

\renewcommand{\thefigure}{\theenumi}
\renewcommand{\thetable}{\theenumi}


\section{ MCQs with One Correct Answer}



\begin{enumerate}
   
\item For a real number $y$, let $\sbrak{y}$ denote the greatest integer less than or equal to $y$. Then the function{$f\brak{x}=\frac{\tan\pi\sbrak{x-\pi}}{1+[x]^2}$} is

\hfill                    (1981 - 2 Marks)
   \begin{enumerate}
       \item discontinuous at some $x$
        \item continuous at all $x$, but the derivative $f^{\prime}(x)$ does not exist for some $x$
        \item $f^{\prime}(x)$ exists for all $x$, but the second derivative $f^{\prime}{\prime}(x)$ does not exist for some $x$
        \item $f^{\prime}(x)$ exists for all $x$
   \end{enumerate}




\item There exists a function $f(x)$,satisfying $f(0)=1, f^{\prime}(0)=-1, f(x)>0$ for all $x$, and 

\hfill      (1982 - 2 Marks)
    \begin{enumerate}

          \item $f^{\prime}(x) > 0$ for all $x$
          \item$ -1 < f^{\prime}{\prime}(x) < 0$ for all $x$
          \item $-2 \leq f^{\prime}{\prime}(x) \leq -1$ for  all 
          \item $f^{\prime}{\prime}(x) < -2$ for all $x$

          
          
           
    \end{enumerate}

\item If $G\brak{x}=-\sqrt{25-x^2}$ then $\lim_{x \to 1}\frac{G(x)-G(1)}{x-1}$ has the value
\hfill    (1983 - 1 Mark)
\begin{enumerate}
    \item $\frac{1}{24}$
    \item $\frac{1}{5}$
    \item -$\sqrt{24}$
    \item none of these
    
\end{enumerate}

\item If $f(a)=2,f^{\prime}(a)=1,g(a)=-1,g^{\prime}(a)=2$, then the value of $\lim_{x\to a}\frac{g(x)f(a)-g(a)f(x)}{x-a}$ is

  \hfill (1983 - 1 Mark)
     \begin{enumerate}
         \item -5
         \item $\frac{1}{5}$
         \item 5
         \item none of these
     \end{enumerate}

\item The function $f(x)=\frac{\ln{(1+ax)}-\ln{(1-bx)}}{x}$ is not defined at $x$=0.The value which should be assigned to $f$ at $x=0$ so that it is continuous at $x$=0, is

  \hfill              (1983 - 1 Mark)
    \begin{enumerate}
        \item a-b
        \item a+b
        \item $\ln{a}-\ln{b}$
        \item none of these
        
    \end{enumerate}

\item $\lim_{n\to \infty}$
      $\left(
      \frac{1}{1-n^2} + \frac{2}{1-n^2} +...+ \frac{n}{1-n^2}
       \right)$ is equal to
       
             \hfill(1984 - 2 Marks)
                 \begin{enumerate}
                    \item 0
                    \item -$\frac{1}{2}$
                    \item $\frac{1}{2}$
                    \item none of these
             
                 \end{enumerate}

\item If $f(x)=\begin{cases}
\frac{\sin\sbrak{x}}{\sbrak{x}}, \sbrak{x}\neq 0\\ 0 ,  \sbrak{x}=0\end{cases}$   where $\sbrak{x}$ denotes the greatest integer less than or equal to $x$, then $\lim_{x\to 0}f(x)$ equals
    \hfill                (1985 - 2 Marks)
    \begin{enumerate}
        \item 1
        \item 0
        \item -1
        \item none of these
    \end{enumerate}

\item Let $f:R\to R$ be a differentiable function and $f(1)=4$.Then the value of
\begin{align}
 \: \lim_{x\to 1}\int_{4}^{f(x)}\frac{2t}{x-1}\,dt 
 \end{align}
 is
\hfill(1990 - 2 Marks)
 \begin{enumerate}
     \item 8$f^{\prime}(1)$
     \item $4f^{\prime}(1)$
     \item $2f^{\prime}(1)$
     \item $f^{\prime}{\prime}(1)$
     
     
 \end{enumerate}

\item Let $\sbrak{.}$ denote the greatest integer function and $f(x)=\sbrak{\tan^2{x}}$, then
\hfill(1993 - 1 Mark)
    \begin{enumerate}
        \item $\lim_{x\to 0}$ does not exist
        \item $f(x)$ is continuous at $x$=0
        \item $f(x)$ is not differentiable at $x$=0
        \item $f^{\prime}(0)=1$
        
    \end{enumerate}

\item The function $f(x)=\sbrak{x}\cos{\brak{\frac{2x-1}{2}}}\pi$ , where $\sbrak{x}$ denotes the greatest integer function, is discontinuous at
\hfill(1995S)
   \begin{enumerate}
       \item All $x$
       \item All integer points
       \item No $x$
       \item $x$ which is not an integer
   \end{enumerate}

\item $\lim_{n\to \infty}\frac{1}{n}\sum_{r=1}^{2n}\frac{r}{\sqrt{n^2+r^2}}$ equals
\hfill(1997 - 2 Marks)
  \begin{enumerate}
      \item 1+$\sqrt{5}$
      \item -1+$\sqrt{5}$
      \item -1+$\sqrt{2}$
      \item 1+$\sqrt{2}$
      
  \end{enumerate}

\item The function $f(x)=\sbrak{x}^2$-$\sbrak{x}^2$ (where $\sbrak{y}$ is the greatest integer less than or equal to $y$), is discontinuous at
\hfill(1992 - 2 Marks)
     \begin{enumerate}
         \item all integers
         \item all integers except 0 and 1
         \item all integers except 0
         \item all integers except 1
         
     \end{enumerate}

\item The function $f(x)=(x^2-1)\abs{x^2-3x+2}+\cos(\abs{x})$ is NOT differentiable at

  \hfill(1999 - 2 Marks)
     \begin{enumerate}
         \item -1
         \item 0
         \item 1
         \item 2
         
     \end{enumerate}

\item $\lim_{x\to0}$$\frac{x\tan(2x)-2x\tan(x)}{(1-\cos(2x))^2}$ is
  \hfill(1999 - 2 Marks)
    \begin{enumerate}
        \item 2
        
        \item -2

        \item $\frac{1}{2}$
        
        \item $\frac{-1}{2}$
        
    \end{enumerate}

\item For $x\in R, \lim_{x\to \infty}(\frac{x-3}{x+2})^x$ =
 \hfill(2000S)
   \begin{enumerate}
       \item $e$
       \item $e^{-1}$
       \item $e^{-5}$
       \item $e^{5}$
       
   \end{enumerate}

\end{enumerate}

\end{document}
